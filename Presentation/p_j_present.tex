%Created by Roland Pastorino in 2013
%Contact info: roland.pastorino@mech.kuleuven.be / www.rolandpastorino.com
%If you want to include your modifications in the template and make them available for everybody, you can contact the "KU Leuven, Dienst Communicatie, Afdeling Positionering en Marketing" (marketing@dcom.kuleuven.be) or myself.

\documentclass[11pt,t]{beamer}
\mode<presentation> {\usetheme{kuleuven}}

% set the way TOC looks
\setbeamertemplate{section in toc}{\inserttocsectionnumber.~\inserttocsection}
\setbeamertemplate{subsection in toc}[default]
\setbeamercolor{section in toc}{fg=black}
\setbeamercolor{subsection in toc}{fg=gray}

\usepackage{animate}
\usepackage{shadowtext}
\usepackage{listings}
\usepackage{xcolor}
\lstset{escapeinside={<@}{@>}}
\usepackage{eurosym}
% set item and enumerate look
\setbeamertemplate{itemize items}[default]
\setbeamertemplate{itemize item}{\color{spot}$\blacktriangleright$}
\setbeamertemplate{itemize subitem}{\color{inthenavy}$\blacksquare$}
\setbeamertemplate{enumerate items}[default]

\InputIfFileExists{defs}{}{} % defs.tex, contains own preamble settings
\tikzset{external/export=false}
\captionsetup[figure]{labelformat=empty}% redefines the caption setup of the figures environment in the beamer class.

%%%%%%%%%%%%%%%%%%%%%%%%%%%%%%%%%%%%%%%%
%info
\title{Scientific Computing with Python and Julia}
%\author{Joseph Szurley, }
%\institute{Group, department, KU Leuven}
%\subtitle{subtitle}
%\date{January 6, 2013, Leuven, Belgium}
%pdf metadata
	\subject{KU Leuven's LaTeX template for oral presentation}
	\keywords{KU Leuven, LaTeX template, beamer, TikZ, pdfLaTeX}
\graphicspath{{graphics/}} % path to the graphics folder
%%%%%%%%%%%%%%%%%%%%%%%%%%%%%%%%%%%%%%%%
\begin{document}
%title page
	\setbeamertemplate{headline}[title_page]
	\setbeamertemplate{footline}[title_page]
	\csname beamer@calculateheadfoot\endcsname %recalculate head and foot dimension
		\begin{frame}
			\titlepage
		\end{frame}
%head and foot for body text	
	\setbeamertemplate{headline}[body]
	\setbeamertemplate{footline}[body]

%%%%%%%%%%%%%%%%%%%%%%%%%%%%%%%%%%%%%%%%%%
%\begin{frame}{Outline}
%	\vskip 5mm
%	\hfill	{\large \parbox{.95\textwidth}{\tableofcontents[hideallsubsections]}}
%\end{frame}

%%%%%%%%%%%%%%%%%%%%%%%%%%%%%%%%%%%%%%%%
%--------------------------------------
\begin{frame}[t]{MATLAB, Python and Julia}
	\begin{columns}[t]
		\begin{column}{.6\textwidth}
		\begin{center}
				\underline{MATLAB}
		\end{center}
		\begin{itemize}
			\item  \textcolor{black}{Closed Source}
			\item \textcolor{black}{Licensing}
				\begin{enumerate}
			\item  MATLAB: 	\euro 2000 (standalone)		
			\item Simulink: \euro 3000 (standalone)	
			\item Added cost for toolboxes
		\end{enumerate}
		\item \textcolor{black}{Currently being phased out of KUL P\&Os} (perhaps more)
		\item  \textcolor{black}{Compatibility}
		\item \textcolor{black}{Strong IDE}
		\item \textcolor{black}{Simulink}
		\end{itemize}
		\end{column}
		\begin{column}{.6\textwidth}
					\begin{center}
				 \underline{Python and Julia}
		\end{center}
		\begin{itemize}
			\item  \textcolor{black}{Open Source}
			\item \textcolor{black}{No cost}
			\item \textcolor{black}{Easy, concise syntax}
				\item \textcolor{black}{Rapid development for testing}
			\item \textcolor{black}{Portability}
		\end{itemize}
		\end{column}
	\end{columns}	
\end{frame}
%--------------------------------------
\begin{frame}[fragile]{Noteworthy Points}
Some similarities between Python and Julia
	\begin{itemize}
	\item Interoperability: can call one from another easily
	\item Interfaces with low level languages (e.g. Cython)
	\item Easily parallelizable
		\end{itemize}
		\vspace{10mm}
		And some differences
			\begin{itemize}
	\item Julia has strong core language, built for scientific computing
	\item Python weak core, relies on third party libraries
   \item Julia has smaller user base
		\end{itemize}
\end{frame}
%--------------------------------------
\begin{frame}[fragile]{Pitfalls}
	\begin{itemize}
	\item Bracket notation: A[i,j] 
	\item Python uses 0-based indexing (Julia is 1-based like MATLAB)
	\small
	\begin{lstlisting}
>>> x = np.array([1, 2])
>>> x[0] 
>>> 1
\end{lstlisting}
\normalsize
		\item {Python and Julia pass by reference (MATLAB passes by value)}
		\small
\begin{lstlisting}
>>> y = x
>>> y[0] = 3
>>> x 
>>> array([3, 2])
\end{lstlisting}
	\item Python allows negative indexing
		\small
\begin{lstlisting}
>>> y[-1]
>>> 2
\end{lstlisting}	
\normalsize
\item Indentation is important in Python!
		\end{itemize}
\end{frame}
%%%%%%%%%%%%%%%%%%%%%%%%%%%%%%%%%%%%%%%

\end{document}